
\section{Введение}

\label{sec:intro}

Информационный текстовый поиск \href{https://en.wikipedia.org/wiki/Information_retrieval}{IR} применяется во многих под-задачах, включая, 
например, поиск нужных документов (параграфов) при генерации ответа через \href{https://en.wikipedia.org/wiki/Large_language_model}{LLM}. 
Так, например, при построении чат-ботов, вопросно-ответных систем и других диалоговых проектов, необходимо выявить релевантные параграфы, которые, с одной стороны,
содержат необходимые ключевые слова, а с другой - являются контекстными (семантически близкими). Проблема усуглубляется тем, что с возрастанием индекса и появлением все большего количества 
параграфов, содержащих похожие ключевые слова, но различающихся по смыслу, приоритетная метрика \href{https://en.wikipedia.org/wiki/Evaluation_measures_(information_retrieval)}{HitRate@k} падает крайне быстро при фиксированном $k=const$. 
Этот эффект во-многом связан с ``проклятием размерности'' (см. \href{https://en.wikipedia.org/wiki/Curse_of_dimensionality}{Curse of dimensionality}) при векторном отображении документов в Евклидово пространство.


Несмотря на  то, что большинство ``старых'' поисковых систем состоят из множества модулей ~(\citet{ferrucci2012introduction,moldovan2003performance}, \textit{inter alia}),
современные диалоговые системы, а также вопросно-ответные системы состоят из двух больших компонентов: \\
(1) \emph{Retriever} - $R(q_i, D) \mapsto \{\d_1, d_2, \cdots, d_k\}$, который 
возвращает релевантные параграфы - $d_1, d_2, \cdots, d_k$, необходимые для $q_i$.
\newline
(2) \emph{LLM} - $L(q_i, R(q, D)) \mapsto a_i$, которая генерирует финальный ответ $a_i$ на вопрос $q_i$, используя выдачу $R(q_i, D)$.\\
В большинстве решений, также, предполагается наличие ранжирующих модулей, однако для простоты изложения, их мы опустим. В случае неточной выдачи $R$ или, наоборот, при отсутствии выдаче вообще, большинство языковых моделей не смогут сгенерировать правильный ответ или, что еще хуже, будут галлюцинировать. ~(\citet{ragsurvey}, \textit{Rag Survey}).

Данная статья посвящена анализу компонента \emph{Retriever} - $R$ и построению алгоритма, позволяющего улучшчить выдачу с соблюдением, как контекстного, так и лексического смысла (пересечение ключевых слов).
При современном проектировании систем, применяется, как правило, векторное отображение $F_\theta$, одной из пред-обученных нейронной моделью, которая впоследствии до-обучается.
Сопосталяемые вектора $\vec{v}_i=F_\theta(d_i) \in \Re^{s}$ добавляются в индекс,  по которому впоследствии выбираются нужные параграфы, используя алгоритм поиска ближайшего соседа \href{https://en.wikipedia.org/wiki/Nearest_neighbor_search}{ANN}.
Для поиска, учитывающего лексический смыл, выражающийся в том, что точные вхождения ключевых слов между запросом и документами играют важную роль, применяются такие алгоритмы, как
TF-IDF или BM25~\cite{robertson2009probabilistic}, которые реализуются через инвертированный индекс.


Гибридный поиск, ранжирующий выдачу между контекстной выдачей и поиском по ключевым словам, путем сглаживания результатов, ввдением параметра $\alpha \in [0..1]$, позволяет улучшить показатели в целом. Однако,
несмотря на ``общие'' результаты - он, по-прежнему справляется плохо на определенных доменах. Как показано в статье ~(\citet{DAPR}, \textit{DAPR}), при комбинации ``легких'' и ``сложных'' запросах $q_i$ - гибридный поиск работает хорошо, но при 
формировании лишь ``сложных'' запросов, где важна контекстность, гибридный поиск работает также плохо.
%


В данной статье исследуется следующая проблема:  можно ли сохранить контекстную полноту выдачи, при условии, что полнота ключевых слов, которые фигурируют между запросом и документами - все еще важна? Как будет показано далее, 
тесты на двух разных наборах данных подтверждают гипотезу о том, что ``гибридный'' работает слишком грубо, пропуская контекстно важные документы, а семантический поиск - не дает приоритет документам, в которых явно выделены ключевые слова, важные для запроса.
Предложенный в данной работе  алгоритм \emph{Atomic Retriever} позволяет улучшить метрику поиска - \href{https://en.wikipedia.org/wiki/Evaluation_measures_(information_retrieval)}{HitRate@2} в среднем на 7\% по сравнению с наилучшими параметрами, как гибридного, так и семантического поиска.
Кроме того, скорость работы в реальном времени по-прежнему остается такой же, как и в случае с поиском ближайшего соседа (семантическим поиском или гибридним). Сложность индексации, однако, возрастает, т.к. помимо сопоставления векторов, необходимо выделять ключевые слова и их объяснение.
