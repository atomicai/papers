
\section{Постановка}
\label{sec:background}

Проблема поиска, изучаемая в данной работе, может быть сформулирована следующим образом. 
Получив на вход вопрос, такой как, например: ``\emph{Какие правила голодных игр?}'' или ``\emph{Какие слова нужно произнести, чтобы открыть карту Мародеров?}'', задача выдать $top_k$ параграфов, 
чтобы среди них содержался нужный. При исследовании данной проблемы, мы ограничиваемся, что на каждый вопрос $q_i$, лишь один параграф $d_i$ правильный. Однако на один 
параграф $d_i$ - возможно несколько вопросов $q^i_1, q^i_2, \cdots, q^i_{\sigma(i)}$


Формально, имеется коллекция документов (параграфов) $D$ \eqref{eq:docs}
\begin{align}
    D=\{d_1, d_2, \cdots, d_{L_D}\}
    \label{eq:docs}
\end{align}

Помимо документов (параграфов) из \eqref{eq:docs} представлен набор запросов $Q$
\begin{align}
    Q=\{q_1, q_2, \cdots, q_{L_Q}\}
    \label{eq:queries}
\end{align}

Стоит отметить, что в данном контексте, на каждый вопрос $q_i$ имеется лишь один праваильный параграф $d_j$, однако на один параграф 
$d_i$ - могут быть несколько правильных вопросов, к которым он относится $d_i \mapsto \{q^j_1, q^j_2, \cdots \}$.

Каждый документ $d_i$ формально можно рассматривать, как последовательность токенов $d_i=\{w^i_1, w^i_2, \cdots, w^i_{l_i}\}$, и, в контексте задачи 
информационного текстового поиска \href{https://en.wikipedia.org/wiki/Information_retrieval}{IR}, предполагается, что необходимо по входящему запросу $q_i$ 
выдать один из существующих $d^i_{+}$. Формально, можно считать, что необходимо создать функцию $R$ - \textit{Retriever} \eqref{eq:retriever}.
\begin{align}
    R=R_k(q_i, D) \mapsto \{d_{\sigma(1)}, d_{\sigma(2)}, \cdots, d_{\sigma(k)}\}
    \label{eq:retriever}
\end{align}

Сам модуль $R$ \eqref{eq:retriever} при фиксированном $k$, которое как правило $k \leq 20$, может быть оценен независимо на разных доменах данных, слеудющим образом.
Пусть $D$ - набор параграфов, а $Q$ - набор вопросов, для ответа на которых нужен один из $d_j \in D$. Правильный параграф для вопроса $q_i$ обозначается как $d^{+}_i$. Тогда \textit{HitRate$@k$} определяется как:
\begin{align}
    \varphi_k(R, {Q, D})=\frac{1}{|Q|}\sum_{i=1}^{|Q|} R_k(q_i, D) \cap d^{+}_i
    \label{eq:varphi}
\end{align}

